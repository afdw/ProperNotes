\documentclass{article}
\usepackage[a4paper, total={19cm, 26cm}]{geometry}
\usepackage[T1]{fontenc}
\usepackage[english]{babel}
\usepackage{amsmath}
\usepackage{amssymb}
\usepackage{amsthm}
\usepackage{mathrsfs}
\usepackage{pifont}
\usepackage{tabu}
\usepackage[pdfusetitle]{hyperref}
\usepackage[dvipsnames, table]{xcolor}
\usepackage{pgfplots}
\usepackage{mathtools}
\usepackage{minted}
\pgfplotsset{compat = 1.17}
\DeclarePairedDelimiter\ceil{\lceil}{\rceil}
\DeclarePairedDelimiter\floor{\lfloor}{\rfloor}
\DeclareMathOperator{\lcm}{lcm}
\DeclareMathOperator{\arsinh}{arsinh}
\DeclareMathOperator{\arcosh}{arcosh}
\DeclareMathOperator{\artanh}{artanh}
\DeclareMathOperator{\DL}{DL}
\DeclareMathOperator{\sign}{sign}
\DeclareMathOperator{\Vect}{Vect}
\DeclareMathOperator{\Ker}{Ker}
\DeclareMathOperator{\nullspace}{null}
\DeclareMathOperator{\range}{range}
\renewcommand{\vec}{\overrightarrow}
\newcommand\restr[2]{{\left.\kern-\nulldelimiterspace #1 \vphantom{\big|} \right|_{#2}}}
\newcommand*\centermathcell[1]{\omit\hfil$\displaystyle#1$\hfil\ignorespaces}
\theoremstyle{plain}\newtheorem{theorem}{Theorem}
\theoremstyle{plain}\newtheorem{proposition}{Proposition}
\theoremstyle{remark}\newtheorem{remark}{Remark}
\newcommand{\backslashbackslash}{\\}
\author{Anton Danilkin}
\title{Null space decomposition for commuting operators}
\begin{document}
\maketitle

Here are some results I do not remember seeing in textbooks (except for \autoref{luS0oah5}), but which seem to be really useful.

\begin{theorem}\label{luS0oah5}
    Let $V$ be a vector space and $S, T \in \mathcal{L}(V)$ such that $S T = T S$. Then $S (\nullspace(T)) \subseteq \nullspace(T)$ and $S (\range(T)) \subseteq \range(T)$.
\end{theorem}
\begin{proof}
    Let $v \in S (\nullspace(T))$; then $v = S(u)$ for some $u \in V$ such that $T(u) = 0$. We have $T(v) = T(S(u)) = S(T(u)) = 0$, so $v \in \nullspace(T)$.

    Let $v \in S (\range(T))$; then $v = S(T(u))$ for some $u \in V$. We have $v = T(S(U))$, so $v \in \range(T)$.
\end{proof}

\begin{theorem}\label{aow7aShu}
    Let $V$ be a finite-dimensional vector space and $S, T \in \mathcal{L}(V)$ such that $S T = T S$ and $\nullspace(S) \cap \nullspace(T) = 0$. Then $S (\nullspace(T)) = \nullspace(T)$.
\end{theorem}
\begin{proof}
    By \autoref{luS0oah5} we already know that $S (\nullspace(T)) \subseteq \nullspace(T)$. We can see that:
    \begin{equation*}
        \begin{aligned}
            S(\nullspace(T)) & = \{ S(v) \mid v \in V, T(v) = 0 \} \\
                             & = \range(\restr{S}{\nullspace{T}})  \\
        \end{aligned}
    \end{equation*}
    And:
    \begin{equation*}
        \begin{aligned}
            0 & = \nullspace(S) \cap \nullspace(T)           \\
              & = \{ v | v \in V, S(v) = 0 \land T(v) = 0 \} \\
              & = \nullspace(\restr{S}{\nullspace{T}})       \\
        \end{aligned}
    \end{equation*}
    But:
    \begin{equation*}
        \dim(\range(\restr{S}{\nullspace{T}})) + \dim(\nullspace(\restr{S}{\nullspace{T}})) = \dim(\nullspace{T})
    \end{equation*}
    So $\dim(S(\nullspace(T))) = 0 + \dim(\nullspace(T))$ and $S (\nullspace(T)) = \nullspace(T)$.
\end{proof}

\begin{theorem}\label{fahch9Wu}
    Let $V$ be a finite-dimensional vector space and $S, T \in \mathcal{L}(V)$ such that $S T = T S$ and $\nullspace(S) \cap \nullspace(T) = 0$. Then $\nullspace(\restr{T}{\range{S}}) = \nullspace(T)$.
\end{theorem}
\begin{proof}
    The direction $\nullspace(\restr{T}{\range{S}}) \subseteq \nullspace(T)$ is easy. Using \autoref{aow7aShu}:
    \begin{equation*}
        \begin{aligned}
            \nullspace(T) & = S (\nullspace(T))                            \\
                          & = \{ S(v) \mid v \in V, T(v) = 0 \}            \\
                          & \subseteq \{ S(v) \mid v \in V, S(T(v)) = 0 \} \\
                          & = \{ S(v) \mid v \in V, T(S(v)) = 0 \}         \\
                          & = \nullspace(\restr{T}{\range{S}})             \\
        \end{aligned}
    \end{equation*}
\end{proof}

\begin{theorem}\label{too2eiGo}
    Let $V$ be a finite-dimensional vector space and $S, T \in \mathcal{L}(V)$ such that $S T = T S$ and $\nullspace(S) \cap \nullspace(T) = 0$. Then $\nullspace(S) \oplus \nullspace(T) = \nullspace(S T)$.
\end{theorem}
\begin{proof}
    The direction $\nullspace(S) \oplus \nullspace(T) \subseteq \nullspace(S T)$ is easy. We can see that:
    \begin{equation*}
        \begin{aligned}
            \range(\restr{T}{\range{S}}) & = \{ T(S(v)) \mid v \in V \} \\
                                         & = \{ S(T(v)) \mid v \in V \} \\
                                         & = \range(S T)                \\
        \end{aligned}
    \end{equation*}
    And $\nullspace(\restr{T}{\range{S}}) = \nullspace(T)$ by \autoref{fahch9Wu}. We have:
    \begin{equation*}
        \begin{aligned}
            \dim(\nullspace(S) \oplus \nullspace(T))
             & = \dim(\nullspace(S)) + \dim(\nullspace(T))                                                                   \\
             & = (\dim(V) - \dim(\range(S))) + (\dim(V) - \dim(\range(T)))                                                   \\
             & = 2 \dim(V) - \dim(\range(S)) - \dim(\range(T))                                                               \\
             & = 2 \dim(V) - (\dim(\range(\restr{T}{\range{S}})) + \dim(\nullspace(\restr{T}{\range{S}}))) - \dim(\range(T)) \\
             & = 2 \dim(V) - (\dim(\range(S T)) + \dim(\nullspace(T))) - \dim(\range(T))                                     \\
             & = (\dim(V) - \dim(\range(S T))) + (\dim(V) - \dim(\range(T)) - \dim(\nullspace(T)))                           \\
             & = \dim(\nullspace(S T))                                                                                       \\
        \end{aligned}
    \end{equation*}
    And so $\nullspace(S) \oplus \nullspace(T) = \nullspace(S T)$.
\end{proof}

These properties (here \autoref{too2eiGo} is taken) could by tested randomly and computationally using a code like the following (using \href{https://github.com/afdw/matrix-operations/blob/main/main.py}{my library for matrix operations}):

\begin{minted}{python}
from random import randint

for _ in range(100000000):
    S = Matrix([[randint(-4, 4), randint(-4, 4), randint(-4, 4)],
                [0,              randint(-4, 4), randint(-4, 4)],
                [0,              0,              randint(-4, 4)]])
    T = Matrix([[randint(-4, 4), randint(-4, 4), randint(-4, 4)],
                [0,              randint(-4, 4), randint(-4, 4)],
                [0,              0,              randint(-4, 4)]])
    if S * T == T * S and \
       S.null().append_right(T.null()).rang() == S.null().m + T.null().m != (S * T).null().m:
        print(S)
        print(T)
\end{minted}

As the linear operators we consider commute, we should always be able to find a common base where the matrices of these operators are triangular, so we only check matrices of this form. Also, coefficients are taken to be fairly small, as non-invertible matrices are rare.
\end{document}
